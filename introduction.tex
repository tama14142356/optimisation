最適化とは
\begin{eqnarray*}
  \Psi (\theta)\ &:&\ 目的関数\\
  \theta\ &:&\ パラメータ
\end{eqnarray*}
としたとき, $\Psi(\theta)$を$\theta$で最小化(あるいは最大化)することである.\ つまり\\
\centerline{\textcolor{red}{\large 関数の最小値, 最大値を与える$x$をもとめる問題}}\\
ということである.
\begin{center}
  \begin{tikzpicture}[>=stealth]
    \draw[->] (-0.7,0)--(4.5,0) node[right] {$x$};
    \draw[->] (0,-0.3)--(0,4.5) node[above] {$y$};
    \draw [domain=-0.3:3.4] plot(\x,{0.5*\x*\x-\x+1.5});
    \draw[draw=red,->,thick] (1,0)--(1,1);
    \draw[dashed] (1,0)--(1,0) node[below] {$x_{0}$};
    \draw[dashed] (1,1)--(0,1) node[left] {$f_{0}$};
    \node at (-0.5,2.2) {$f(x)$};
  \end{tikzpicture}
\end{center}
  このときの
\begin{eqnarray*}
    f_{0}&=&\underset{x}{\rm min}\,f(x)\\
    x_{0}&=&\underset{x}{\rm argmin}\,f(x)
\end{eqnarray*}
を求める問題である.
$\theta$が関数の形状を決めるパラメータであったとき, 目的関数$\Psi(\theta)$は近似誤差を表す関数となり, 最適化は関数$f(x;\theta)$の形状を求める問題といえる.